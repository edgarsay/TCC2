\vfill
\begin{center}
{\textbf{ABSTRACT}\\}
\end{center}

\noindent

Games are a very common tool in education, but the development of this tool can still be a difficult and confusing process that raises doubts as to which technology is the most suitable for a certain context. This article shows the process of development and working of a \textit{framework} for creating educational Web games with Canvas and SVG. Taking advantage of the opportunity provided by this tool, it is possible to make a comparison of both technologies in several cases. The main contribution of this article is to facilitate the creation of educational games, more specifically board games in the Portuguese language. The Working of this tool will guide game developers to use this tool in their creations. Finally, the comparison of the performance of Canvas and SVG will serve both those who wish to create their own Web game frameworks, developers interested in using this framework, which allows the use of both technologies, to choose which solution fits best in your games and  also can be useful to anyone who has questions.
 
 \vspace{\onelineskip}
    
 \noindent
 \textbf{Keywords}: First. Second. Third.
