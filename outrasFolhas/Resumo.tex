\vfill
\begin{center}
{\textbf{RESUMO}\\}
\end{center}
\noindent

Os jogos são uma ferramenta muito comum na educação, porém o desenvolvimento dessa ferramenta ainda pode ser um processo difícil e confuso que cria dúvidas de qual a melhor tecnologia a ser utilizada. Este artigo apresenta o processo de desenvolvimento e funcionamento  de um \textit{framework} para a criação de jogos educativos na Web com Canvas e SVG. Aproveitando a oportunidade disponibilizada por esta ferramenta torna-se possível fazer um comparativo de ambas tecnologias em diversos casos. A principal contribuição deste artigo é facilitar a criação de jogos educativos, mais especificamente jogos de tabuleiro na língua portuguesa. O funcionamento dessa ferramenta servirá para orientar desenvolvedores de jogos a usar este instrumento em suas criações. Por último, o comparativo que será feito do desempenho do Canvas e SVG servirá tanto àqueles que desejam criar seus próprios \textit{frameworks} de desenvolvimento Web, como também desenvolvedores interessados em usar este \textit{framework}, que  permite o uso de ambas tecnologias, a escolher qual solução se encaixa melhor em seus jogos e pode ser útil para qualquer outro que possua dúvidas entre o uso destas ferramentas.

\vspace{\onelineskip}
 \noindent
 \textbf{Palavras-chaves}: Primeira. Segunda. Terceira.
