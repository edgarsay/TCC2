\chapter{INTRODUÇÃO}
\label{Introducao}

\section{Motivação}

%#1
A educação é um componente essencial da sociedade em que vivemos, este é um fator determinante para o desenvolvimento humano. Portanto a qualidade da educação disponível em um país é um forte indicativo do seu futuro, melhorando-o ou piorando-o. De acordo como  Global Partnership for Education, em um \textit{rank} educacional, o Brasil está em 55º com 528 pontos de distância do primeiro colocado. O Brasil possui 12 milhões de analfabetos e mais 50\% dos adultos entre 25 e 64 anos não concluíram o Ensino médio.

Atualmente os métodos de ensino tradicionais têm dificuldades para engajar e estimular seus estudantes. O uso de jogos como ferramenta auxilia no ensino e muitas vezes proporcionam resultados melhores \cite{Girard&Ecalle&Magnan:13}. Essa aplicação pode se apresentar através da \textit{gamification}, o uso de mecânicas e dinâmicas de jogos para engajar pessoas. Assim, para resolver problemas e melhorar o aprendizado ,  onde alguns aspectos base dos videogames são adaptados para outros cenários, como os pontos que  são uma estrutura básica em jogos \cite{Zichermann&Cunningham:11} usado com propósito de motivar o jogador e também mostrar o seu progresso na tarefa que está sendo realizada.

Brincar ou jogar é uma atividade natural durante a vida de inúmeros animais, incluindo o ser humano, principalmente durante a infância. Essa atividade naturalmente atrai os seres humanos e é fruto de um processo evolutivo. Os animais que jogam ou simulam atividades antes de realizá-las de fato possuem um desempenho melhor. Por tanto aprender brincando é o método natural de adquirir e reforçar o conhecimento \cite{Bekoff&DiMotta:08}.

Alguns professores podem interpretar jogos como um inimigo às suas aulas, porém os mesmos podem se tornar poderosos aliados. A assistência de computadores na educação é conhecida como uma ferramenta efetiva para melhorar o aprendizado tanto em adultos quanto em crianças, pois pode ser muito eficiente em prover engajamento nos estudos \cite{Girard&Ecalle&Magnan:13}.

%#2
Os jogos usados como ferramentas geralmente são chamados \textit{serios games}. Estes são capazes de aumentar o engajamento dos alunos, dando aos mesmos a possibilidade de aprender conteúdo teórico de forma divertida e no seu próprio ritmo, melhorando a qualidade do aprendizado em uma área específica. Existem diversas soluções para a criação de jogos, porém estas raramente são especificamente focadas em jogos educativos e não costumam ser amigáveis, principalmente para não falantes da língua inglesa.

Como \cite{liu2013effect} em [\textit{ The Effect of Game-Based Learning on Students}] mostram um estudo feito com crianças do jardim de infância. Os alunos que tiveram uma aula com auxílio de \textit{gamification}, como jogos e aplicativos, mostraram resultados melhores. Portanto \textit{gamefication} pode ser um poderoso aliado, mas criar ferramentas educativas pode se mostra um desafio \cite{liu2013effect}.

A Web é uma plataforma muito comum para jogos educativos, pois apesar de ainda não possuírem a
mesma capacidade gráfica que os jogos nativos, possuem vantagem em distribuição. Qualquer um
pode ter acesso usando o dispositivo e sistema operacional que achar mais confortável, esse 
aspecto é muito mais importante que eficiência no que diz respeito à educação, além de permitir
atualizações sem nenhum esforço do usuário \cite{Mills:19}.

Desenvolver um jogo de tabuleiro para \textit{Web} pode se tornar muito complexo, principalmente para pessoas inexperientes. Para auxiliar no desenvolvimento de jogos educativos é preferível o uso de ferramentas como \textit{frameworks}, que são capazes de facilitar a criação do aplicativo. Estes \textit{frameworks} devem criar automaticamente as partes comuns do projeto, deixando assim que o criador faça apenas aquilo que é necessário, para que o jogo se ajuste as necessidades da matéria a ser ensinada. Por consequência a ferramenta permitirá ao usuário criar conteúdo original, interativo, dinâmico. Sem \textit{frameworks} todos esse aspectos se tornam, muitas vezes, difíceis de serem implementados. Portanto um \textit{framework} contendo partes que pudessem ser desenvolvidas com uma ferramenta de software, além da programação de código,seria muito mais desejável ainda.

%#TODO
%Adicionar então: 
%(4) Delimitação do tema → jogos de tabuleiro que possam ser usados para ensino e diversão, usando sistemas computacionais leves (isto é, não precisa de CPU e GPU de última geração!).
%(5) Fazer comentário sobre objetivo de ser fácil e ter além do framework/código uma ferramenta:

\section{Objetivos}

Nesta secção serão abordados o objetivos, almejado neste trabalho, de duas forma objetivo geral e específicos.

\subsection{Objetivo Geral}

O trabalho pretende fornecer um \textit{framework} para desenvolvimento de jogos educativos leves de tabuleiro com gráficos 2D usando tecnologia Web. O \textit{framework} envolverá tanto uma base de código como ferramentas de software adicionais para facilitar o seu uso.


\subsection{Objetivos Específicos}

%#TODO:
%(1) Introduzir os principais conceitos teóricos relacionados a um framework de jogos gráficos 2D;(OBS: ver seção 2.1. mais adiante pra essa parte – que você vai fazer)

%(2) Introduzir os principais conceitos tecnológicos relacionados a um framework de jogos gráficos 2D;(OBS: ver seção 2.2 mais adiante para essa parte – que você já fez)

%(3) Apresentar uma investigação sobre frameworks relacionados de jogos para Web;(OBS: isso você já fez: Phaser, PixiJS, Kiwi.js, CreateJS, Construct, iLearnTest, análise geral; isso tudo ficará na Seção 3)

%(4) Implementar o framework para jogos educativos de tabuleiro (código);(OBS: isso já tem sido feito ao longo do tempo; precisa agora colocar no TCC trechos do seu código)

%(5) Implementar as ferramentas de software do framework;(6) Implementar um exemplo prático onde se aplica o framework, demonstrando assim suascapacidades;(OBS: isso já tem sido feito ao longo do tempo; você pode colocar screenshots do exemplo (demo)com breves explicações das telas e um pouco do código que utilizou pra o demo]


Esse objetivo aqui almejado deve ser alcançado através de um ferramenta simples e intuitiva, que seja capaz de entregar jogos de tabuleiro, como o mini de esforço por parte do desenvolvedor. Aquele que optar pela uso deste mecanismo, deverá desfrutar de um processo de criação onde fora suas preocupações apenas no conteúdo educacional que desejar apresentar através do jogo, pois o desenvolvimento será feito baseando-se em templates, logo o criador não terá maiores preocupações com as mecânicas do jogo. Por fim, se necessário ou desejável o usuário deste equipamento poderá modificar parte central da tecnologia, portanto podendo descomplicadamente, com pouca alterações em seu código, fazer uso de Canvas ou SVG em sua aplicação.

\section{Organização do Trabalho}

O restante desse trabalho é organizado da seguinte forma: na Seção 2 os conceitos teóricos e tecnológicosserão apresentados; na Seção 3, uma investigação sobre frameworks da área é apresentada; na Seção 4, aproposta de framework do presente trabalho é descrita; na Seção 5, as ferramentas do framework sãodescritas; na Seção 6, um exemplo (demo) é apresentado para ilustrar as capacidades do frameworkapresentado, com resultados e discussões; a Seção 7 apresenta as conclusões e opções de trabalhosfuturos.