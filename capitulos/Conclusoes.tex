\chapter{Conclusão e Trabalhos Futuros}

Pesquisadores que atuam na área de educação estão sempre buscando um caminho contínuo de melhorias para  o  processo educacional.  Várias propostas  são feitas buscando mediar  um ensino mais  rico e produtivo   tanto   para   alunos   quanto   professores.   A   ideia   de   jogos   educativos   tem   se   estabelecido firmemente no cenário atual. Assim, o presente trabalho buscou se inserir dentro desse contexto, visando agregar mais opções para esse campo de atuação.

O objetivo geral do trabalho foi fornecer um esquema de desenvolvimento de jogos educativos de tabuleiro com tecnologias Web chamado eJET. O uso da plataforma Web para o trabalho permite um grande alcance de usuários; além disso, buscou-se um tipo de implementação que permitisse uma execução de jogos leves, que não exigissem hardware mais especializado. O processo de pesquisa foi conduzido primeiramente apresentando-se os principais conceitos teóricos da área e em seguida os conceitos   tecnológicos   envolvidos   na   implementação.   A   pesquisa,   em   seguida,   apresentou   uma investigação dos principais \textit{frameworks} disponíveis para a tarefa. Essa preparação conceitual-comparativa serviu finalmente de base para o próprio projeto sendo proposto, que envolveu tanto um \textit{frameworks} para programação com código quanto ferramentas (software) adicionais. Isso foi feito assim para cumprir os objetivos delineados no início do trabalho em relação à facilidade, a abrangência, o aproveitamento de estruturas regulares para automatização e o incentivo para mais pessoas utilizarem o eJET. Este próprio trabalho serve como introdução ao eJET para entusiastas na Internet, contendo tanto os conceitos quanto as instruções práticas, e adotando ainda o modelo de Software Livre para difusão aberta da tecnologia, convidando a todos para usar, contribuir e expandir o eJET.

Para trabalhos futuros, espera-se expandir o sistema desenvolvido aqui para incorporar partidas \textit{on-line} entre   jogadores  via   Internet   e   a  adequação  mais   automatizada   para   dispositivos  móveis,   além   da investigação de gráficos com WebGL para crescimento em termos de opções visuais.

\label{Conclusao}
